% Options for packages loaded elsewhere
% Options for packages loaded elsewhere
\PassOptionsToPackage{unicode}{hyperref}
\PassOptionsToPackage{hyphens}{url}
%
\documentclass[
  ngerman,
  letterpaper,
]{scrbook}
\usepackage{xcolor}
\usepackage[paperwidth=7in,paperheight=10in]{geometry}
\usepackage{amsmath,amssymb}
\setcounter{secnumdepth}{5}
\usepackage{iftex}
\ifPDFTeX
  \usepackage[T1]{fontenc}
  \usepackage[utf8]{inputenc}
  \usepackage{textcomp} % provide euro and other symbols
\else % if luatex or xetex
  \usepackage{unicode-math} % this also loads fontspec
  \defaultfontfeatures{Scale=MatchLowercase}
  \defaultfontfeatures[\rmfamily]{Ligatures=TeX,Scale=1}
\fi
\usepackage{lmodern}
\ifPDFTeX\else
  % xetex/luatex font selection
  \setmainfont[]{Lora}
  \setsansfont[]{Lato}
  \setmonofont[]{Roboto}
\fi
% Use upquote if available, for straight quotes in verbatim environments
\IfFileExists{upquote.sty}{\usepackage{upquote}}{}
\IfFileExists{microtype.sty}{% use microtype if available
  \usepackage[]{microtype}
  \UseMicrotypeSet[protrusion]{basicmath} % disable protrusion for tt fonts
}{}
\makeatletter
\@ifundefined{KOMAClassName}{% if non-KOMA class
  \IfFileExists{parskip.sty}{%
    \usepackage{parskip}
  }{% else
    \setlength{\parindent}{0pt}
    \setlength{\parskip}{6pt plus 2pt minus 1pt}}
}{% if KOMA class
  \KOMAoptions{parskip=half}}
\makeatother
% Make \paragraph and \subparagraph free-standing
\makeatletter
\ifx\paragraph\undefined\else
  \let\oldparagraph\paragraph
  \renewcommand{\paragraph}{
    \@ifstar
      \xxxParagraphStar
      \xxxParagraphNoStar
  }
  \newcommand{\xxxParagraphStar}[1]{\oldparagraph*{#1}\mbox{}}
  \newcommand{\xxxParagraphNoStar}[1]{\oldparagraph{#1}\mbox{}}
\fi
\ifx\subparagraph\undefined\else
  \let\oldsubparagraph\subparagraph
  \renewcommand{\subparagraph}{
    \@ifstar
      \xxxSubParagraphStar
      \xxxSubParagraphNoStar
  }
  \newcommand{\xxxSubParagraphStar}[1]{\oldsubparagraph*{#1}\mbox{}}
  \newcommand{\xxxSubParagraphNoStar}[1]{\oldsubparagraph{#1}\mbox{}}
\fi
\makeatother


\usepackage{longtable,booktabs,array}
\usepackage{calc} % for calculating minipage widths
% Correct order of tables after \paragraph or \subparagraph
\usepackage{etoolbox}
\makeatletter
\patchcmd\longtable{\par}{\if@noskipsec\mbox{}\fi\par}{}{}
\makeatother
% Allow footnotes in longtable head/foot
\IfFileExists{footnotehyper.sty}{\usepackage{footnotehyper}}{\usepackage{footnote}}
\makesavenoteenv{longtable}
\usepackage{graphicx}
\makeatletter
\newsavebox\pandoc@box
\newcommand*\pandocbounded[1]{% scales image to fit in text height/width
  \sbox\pandoc@box{#1}%
  \Gscale@div\@tempa{\textheight}{\dimexpr\ht\pandoc@box+\dp\pandoc@box\relax}%
  \Gscale@div\@tempb{\linewidth}{\wd\pandoc@box}%
  \ifdim\@tempb\p@<\@tempa\p@\let\@tempa\@tempb\fi% select the smaller of both
  \ifdim\@tempa\p@<\p@\scalebox{\@tempa}{\usebox\pandoc@box}%
  \else\usebox{\pandoc@box}%
  \fi%
}
% Set default figure placement to htbp
\def\fps@figure{htbp}
\makeatother
\usepackage{svg}


% definitions for citeproc citations
\NewDocumentCommand\citeproctext{}{}
\NewDocumentCommand\citeproc{mm}{%
  \begingroup\def\citeproctext{#2}\cite{#1}\endgroup}
\makeatletter
 % allow citations to break across lines
 \let\@cite@ofmt\@firstofone
 % avoid brackets around text for \cite:
 \def\@biblabel#1{}
 \def\@cite#1#2{{#1\if@tempswa , #2\fi}}
\makeatother
\newlength{\cslhangindent}
\setlength{\cslhangindent}{1.5em}
\newlength{\csllabelwidth}
\setlength{\csllabelwidth}{3em}
\newenvironment{CSLReferences}[2] % #1 hanging-indent, #2 entry-spacing
 {\begin{list}{}{%
  \setlength{\itemindent}{0pt}
  \setlength{\leftmargin}{0pt}
  \setlength{\parsep}{0pt}
  % turn on hanging indent if param 1 is 1
  \ifodd #1
   \setlength{\leftmargin}{\cslhangindent}
   \setlength{\itemindent}{-1\cslhangindent}
  \fi
  % set entry spacing
  \setlength{\itemsep}{#2\baselineskip}}}
 {\end{list}}
\usepackage{calc}
\newcommand{\CSLBlock}[1]{\hfill\break\parbox[t]{\linewidth}{\strut\ignorespaces#1\strut}}
\newcommand{\CSLLeftMargin}[1]{\parbox[t]{\csllabelwidth}{\strut#1\strut}}
\newcommand{\CSLRightInline}[1]{\parbox[t]{\linewidth - \csllabelwidth}{\strut#1\strut}}
\newcommand{\CSLIndent}[1]{\hspace{\cslhangindent}#1}

\ifLuaTeX
\usepackage[bidi=basic]{babel}
\else
\usepackage[bidi=default]{babel}
\fi
\ifPDFTeX
\else
\babelfont{rm}[]{Lora}
\fi
% get rid of language-specific shorthands (see #6817):
\let\LanguageShortHands\languageshorthands
\def\languageshorthands#1{}
\ifLuaTeX
  \usepackage[german]{selnolig} % disable illegal ligatures
\fi


\setlength{\emergencystretch}{3em} % prevent overfull lines

\providecommand{\tightlist}{%
  \setlength{\itemsep}{0pt}\setlength{\parskip}{0pt}}



 

\usepackage[]{csquotes}

% load packages
%\usepackage{multicol}
\usepackage{fontspec}
\usepackage{emoji}
\usepackage{xltxtra}
%\usepackage{xcolor}
\usepackage{listings}
\usepackage{fvextra}
%\usepackage[german, provide=*]{babel}
\usepackage{csquotes}
\usepackage{tocloft}  % more space in TOC
\usepackage{microtype}


\definecolor{ycol}{RGB}{230,159,0}
\definecolor{modelcol}{RGB}{86,180,233}
\definecolor{errorcol}{RGB}{0,158,115}
\definecolor{beta0col}{RGB}{213,94,0}
\definecolor{beta1col}{RGB}{0,114,178}
\definecolor{xcol}{RGB}{204,121,167}


% Verbesserte Silbentrennung
\lefthyphenmin=2
\righthyphenmin=3
\hyphenpenalty=50
\exhyphenpenalty=50

% Für sehr lange Wörter
\tolerance=1000
\emergencystretch=3em
\hbadness=10000


\lstset{
  breaklines=true
}

\DefineVerbatimEnvironment{Highlighting}{Verbatim}{breaklines,commandchars=\\\{\}}
\DefineVerbatimEnvironment{OutputCode}{Verbatim}{breaklines,commandchars=\\\{\}}


\raggedbottom
\flushbottom


\setlength{\cftchapnumwidth}{4em} % Increase this to give more room to the number



% rm page number of "part pages", added 2025-04-17
\makeatletter
\def\@part[#1]#2{%
  \ifnum \c@secnumdepth > -1
    \refstepcounter{part}%
    \addcontentsline{toc}{part}{\thepart\hspace{1em}#1}%
  \else
    \addcontentsline{toc}{part}{#1}%
  \fi
  \markboth{}{}%
  \clearpage
  \thispagestyle{empty} % <-- removes page number
  {\centering
    \interlinepenalty \@M
    \normalfont
    \ifnum \c@secnumdepth > -1
      \huge\bfseries \partname~\thepart
      \par\nobreak
      \vskip 20pt
    \fi
    \Huge \bfseries #2\par
    \vskip 40pt
  }
  \@afterheading
}
\makeatother


\usepackage{booktabs}
\usepackage{caption}
\usepackage{longtable}
\usepackage{colortbl}
\usepackage{array}
\usepackage{anyfontsize}
\usepackage{multirow}
\makeatletter
\@ifpackageloaded{tcolorbox}{}{\usepackage[skins,breakable]{tcolorbox}}
\@ifpackageloaded{fontawesome5}{}{\usepackage{fontawesome5}}
\definecolor{quarto-callout-color}{HTML}{909090}
\definecolor{quarto-callout-note-color}{HTML}{0758E5}
\definecolor{quarto-callout-important-color}{HTML}{CC1914}
\definecolor{quarto-callout-warning-color}{HTML}{EB9113}
\definecolor{quarto-callout-tip-color}{HTML}{00A047}
\definecolor{quarto-callout-caution-color}{HTML}{FC5300}
\definecolor{quarto-callout-color-frame}{HTML}{acacac}
\definecolor{quarto-callout-note-color-frame}{HTML}{4582ec}
\definecolor{quarto-callout-important-color-frame}{HTML}{d9534f}
\definecolor{quarto-callout-warning-color-frame}{HTML}{f0ad4e}
\definecolor{quarto-callout-tip-color-frame}{HTML}{02b875}
\definecolor{quarto-callout-caution-color-frame}{HTML}{fd7e14}
\makeatother
\makeatletter
\@ifpackageloaded{caption}{}{\usepackage{caption}}
\AtBeginDocument{%
\ifdefined\contentsname
  \renewcommand*\contentsname{Inhaltsverzeichnis}
\else
  \newcommand\contentsname{Inhaltsverzeichnis}
\fi
\ifdefined\listfigurename
  \renewcommand*\listfigurename{Abbildungsverzeichnis}
\else
  \newcommand\listfigurename{Abbildungsverzeichnis}
\fi
\ifdefined\listtablename
  \renewcommand*\listtablename{Tabellenverzeichnis}
\else
  \newcommand\listtablename{Tabellenverzeichnis}
\fi
\ifdefined\figurename
  \renewcommand*\figurename{Abbildung}
\else
  \newcommand\figurename{Abbildung}
\fi
\ifdefined\tablename
  \renewcommand*\tablename{Tabelle}
\else
  \newcommand\tablename{Tabelle}
\fi
}
\@ifpackageloaded{float}{}{\usepackage{float}}
\floatstyle{ruled}
\@ifundefined{c@chapter}{\newfloat{codelisting}{h}{lop}}{\newfloat{codelisting}{h}{lop}[chapter]}
\floatname{codelisting}{Listing}
\newcommand*\listoflistings{\listof{codelisting}{Listingverzeichnis}}
\makeatother
\makeatletter
\makeatother
\makeatletter
\@ifpackageloaded{caption}{}{\usepackage{caption}}
\@ifpackageloaded{subcaption}{}{\usepackage{subcaption}}
\makeatother
\usepackage{bookmark}
\IfFileExists{xurl.sty}{\usepackage{xurl}}{} % add URL line breaks if available
\urlstyle{same}
\hypersetup{
  pdftitle={Start:Bayes!},
  pdfauthor={Sebastian Sauer},
  pdflang={de},
  hidelinks,
  pdfcreator={LaTeX via pandoc}}


\title{Start:Bayes!}
\author{Sebastian Sauer}
\date{2025-09-03}
\begin{document}
\frontmatter
\maketitle


\mainmatter
\chapter{Einführung}\label{einfuxfchrung}

\begin{figure}[H]

{\centering \includegraphics[width=0.5\linewidth,height=\textheight,keepaspectratio]{img/Golem_hex.png}

}

\caption{Bayes:Start! Bildquelle: Klara Schaumann}

\end{figure}%

\section{Ihr Lernerfolg}\label{ihr-lernerfolg}

\subsection{Lernziele}\label{lernziele}

Nach diesem Kurs sollten Sie \ldots{}

\begin{itemize}
\tightlist
\item
  grundlegende Konzepte der Inferenzstatistik mit Bayes verstehen und
  mit R anwenden können
\item
  gängige einschlägige Forschungsfragen in statistische Modelle
  übersetzen und mit R auswerten können
\item
  kausale Forschungsfragen in statistische Modelle übersetzen und prüfen
  können
\item
  die Güte und Grenze von statistischen Modellen einschätzen können
\end{itemize}

\subsection{Was lerne ich hier und wozu ist das
gut?}\label{was-lerne-ich-hier-und-wozu-ist-das-gut}

\emph{Kurz gesagt, warum soll ich das lernen?}

Statistische Analysen sind die Grundlage für Entscheidungen: Nehmen wir
zum Beispiel an, Sie haben Sie 50 Frauen und Männer vor eine
Einpark-Aufgabe gestellt (natürlich alles schön standardisiert und
kontrolliert) - Wer am schnellsten ein Auto einparken kann. Das
Ergebnis: Frauen können schneller einparken als Männer, im Durchschnitt.
Das hätten wir also geklärt. Aber haben wir das ganz sicher geklärt? Mit
welcher Sicherheit? Bekanntlich sind in dieser Welt nur Steuern und der
Tod sicher; sonstige Aussagen leider nicht und damit unsere
Einpark-Studie und sonstige statistische Analysen auch nicht. Ja, ich
weiß, das ist jetzt ein harter Schlag für Sie\ldots{} Aber die gute
Nachricht ist: Wir können angeben, wie (un)sicher wir bei mit einer
Aussage (\enquote{Frauen parken schneller\ldots{}}) sind. Zum Beispiel
könnten wir uns zu 99\% oder zu 51\% sicher sein - und \emph{wie sicher}
wir uns sind, macht schon einen Unterschied. Wenn Sie nächste Woche ei
Fahri für Ihren neuen Rolls Royce anheuern, müssen Sie ja wissen, ob es
besser eine Frau oder ein Mann sein soll.

Kurz gesagt: In diesem Kurs lernen Sie, wie Sie die Unsicherheit eines
statistischen Ergebnisses beziffern.

\emph{Warum ist das wichtig?}

Da fast keine Aussage auf dieser Welt 100\% sicher ist, müssen wir
wissen, wie sicher eine Aussage ist, wenn wir eine Entscheidung treffen
wollen.

\emph{Wozu brauche ich das im Job?}

Ihr Boss wird wissen wollen, wie sicher Sie sich sind, wenn Sie sagen
\enquote{laut meiner Analyse sollten wir unser Werk in
Ansbach/Peking/Timbuktu bauen}. Sind Sie sich zu 50\%, 90\% oder 99,9\%
sicher, dass Ihre Aussage richtig ist? Wichtige Frage im echten Leben.

\emph{Wozu brauche ich das im weiterem Studium?}

In Forschungsarbeiten (wie in empirischen Forschungsprojekten, etwa in
der Abschlussarbeit) ist es üblich, statistische Ergebnisse hinsichtlich
ihrer Unsicherheit zu beziffern.

\emph{Gibt es auch gute Jobs, wenn man sich mit Daten auskennt?}

Das Forum (2020) berichtet zu den \enquote{Top 20 job roles in
increasing and decreasing demand across industries} (S. 30, Abb. 22):

\begin{enumerate}
\def\labelenumi{\arabic{enumi}.}
\tightlist
\item
  Data Analysts und Scientists
\item
  AI and Machine Learning Specialists
\item
  Big Data Specialists
\end{enumerate}

\subsection{Voraussetzungen}\label{voraussetzungen}

Für dieses Kurs wird folgendes Wissen vorausgesetzt:

\begin{itemize}
\tightlist
\item
  grundlegende Kenntnis im Umgang mit R, möglichst auch mit dem
  \texttt{tidyverse}
\item
  grundlegende Kenntnis der deskriptiven Statistik
\item
  grundlegende Kenntnis der Regressionsanalyse
\end{itemize}

Dieses Wissen wird z.B. im
\href{https://statistik1.netlify.app/}{Online-Buch \enquote{Statistik1}}
vermittelt. Alle Inhalte daraus werden in diesem Kurs benötigt.

\subsection{PDF-Version}\label{pdf-version}

Sie können die Druck-Funktion Ihres Broswers nutzen, um ein PDF-Dokument
eines Kapitels dieses Buchs zu erstellen.

Alternativ finden Sie
\href{https://github.com/sebastiansauer/start-bayes/tree/main/pdf}{hier}
die Kapitel als PDF-Version. Achtung: Diese PDF-Versionen sind nicht
unbedingt aktuell.

\section{Lernhilfen}\label{lernhilfen}

\href{https://hinweisbuch.netlify.app/160-hinweise-lernhilfen-frame}{Hier}
finden Sie einen Überblick zu Lernhilfen.

\section{Software}\label{software}

Sie benötigen R, RStudio und einige R-Pakete insbesondere
\texttt{rstanarm} für diesen Kurs.

\href{https://hinweisbuch.netlify.app/130-hinweise-software}{Hier}
finden Sie \emph{Installationshinweise.}

\section{Hinweise}\label{hinweise}

\begin{tcolorbox}[enhanced jigsaw, opacitybacktitle=0.6, breakable, titlerule=0mm, colbacktitle=quarto-callout-note-color!10!white, rightrule=.15mm, toptitle=1mm, opacityback=0, bottomtitle=1mm, title=\textcolor{quarto-callout-note-color}{\faInfo}\hspace{0.5em}{Hinweis}, arc=.35mm, coltitle=black, bottomrule=.15mm, colframe=quarto-callout-note-color-frame, left=2mm, leftrule=.75mm, toprule=.15mm, colback=white]

Alle formalen Hinweise (Prüfung, Unterrichtsorganisation, \ldots) sind
auf der Seite \url{https://hinweisbuch.netlify.app/} zu finden.
\(\square\)

\end{tcolorbox}

\begin{itemize}
\item
  📺
  \href{https://www.youtube.com/watch?v=QNMVi6IqQ90&list=PLRR4REmBgpIGgz2Oe2Z9FcoLYBDnaWatN}{Playlist
  QM2})
\item
  \href{https://hinweisbuch.netlify.app/160-hinweise-lernhilfen-frame}{Lernhilfen}
\item
  \href{https://hinweisbuch.netlify.app/110-hinweise-didaktik-frame}{Didaktik}
\item
  \href{https://hinweisbuch.netlify.app/120-hinweise-unterricht-frame}{Unterrichtsorganisation}
\item
  Der Unterricht zu diesem Modul wird nur ein Mal pro Jahr angeboten
  (also nur jedes zweite Semester).
\item
  Eine Prüfung in diesem Modul ist jedes Semester möglich.
\end{itemize}

\section{Tutorium}\label{tutorium}

Für dieses Modul wird ggf. ein Tutorium angeboten.

Der Besuch des Tutoriums ist zu empfehlen. Arbeiten Sie auch das
Materials auf der \href{https://qm2-tutorium.netlify.app/}{Webseite des
Tutoriums} durch.

\section{Prüfung}\label{pruxfcfung}

Das aktuelle Prüfungsformat ist: \emph{Klausur im Mehrfachwahlverfahren
(Multiple Choice)}.

Hilfsmittel wie Skripte oder Notizen sind nicht zulässig. Die Prüfung
findet (ausschließlich) in Präsenz statt.

\begin{itemize}
\tightlist
\item
  \href{https://hinweisbuch.netlify.app/010-hinweise-pruefung-allgemein-frame}{Allgemeine
  Prüfungshinweise}
\end{itemize}

\begin{itemize}
\item
  \href{https://hinweisbuch.netlify.app/030-hinweise-pruefung-klausur-frame}{Hinweise
  zu quantitativen Prüfungen}
\item
  \href{https://hinweisbuch.netlify.app/030-hinweise-pruefung-klausur-frame\#ablauf-einer-pr\%C3\%BCfung}{Hinweise
  zum Ablauf von Klausuren}
\item
  \href{https://hinweisbuch.netlify.app/150-hinweise-pruefungsvorbereitung-frame}{Prüfungsvorbereitung}
\end{itemize}

In \textbf{?@sec-abschluss} finden sich weitere Hinweise auch mit Blick
zu Aufgabensammlungen.

\section{Zitation}\label{zitation}

Bitte zitieren Sie dieses Buch wie folgt:

\begin{displayquote}
Sauer, S. (2023). \emph{Start:Bayes!}. https://start-bayes.netlify.app/
\end{displayquote}

Hier sind die maschinenlesbaren Zitationsinfos (Bibtex-Format), die Sie
in Ihre Literatursoftware importieren können:

\begin{verbatim}
@book{sauer_startbayes,
    title = {Start:Bayes},
    rights = {CC-BY-NC},
    url = {https://start-bayes.netlify.app/},
    author = {Sauer, Sebastian},
    date = {2025},
}
\end{verbatim}

Hier ist die DOI:

\begin{figure}[H]

{\centering \pandocbounded{\includesvg[keepaspectratio]{index_files/mediabag/532213155.svg}}

}

\caption{DOI}

\end{figure}%

\section{Zum Autor}\label{zum-autor}

Nähere Hinweise zum Autor, Sebastian Sauer, finden Sie
\href{https://sebastiansauer-academic.netlify.app/}{hier}.

\part{Organisatorisches}

\phantomsection\label{refs}
\begin{CSLReferences}{1}{0}
\bibitem[\citeproctext]{ref-forum2020}
Forum, W. E. (2020). \emph{The {Future} of {Jobs Report} 2020}. World
Economic Forum.
\url{https://www3.weforum.org/docs/WEF_Future_of_Jobs_2020.pdf}

\end{CSLReferences}


\backmatter


\end{document}
